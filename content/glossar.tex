\begin{glossar}[Prioritätenskala] % Das Muster dient zur Bestimmung der Einrueckungstiefe
    \item[DInf]             Departement Informatik
    \item[FFHS]             Fernfachhochschule Schweiz
    %\item[Prioritätenskala] Die Prioritätenskala ist innerhalb des Projektes stets mit MUSS, SOLL, KANN gekennzeichnet. Dabei gilt die hier definierte ordinale Rangfolge.
	%\item[Client]           Gemeint ist jeweils das Front-End. Mit Front-End sind innerhalb dieses Projektes alle Teile gemeint, die primär für die Darstellung, Ausgabe sowie Eingabe von Daten über das GUI genutzt werden.
    %\item[Server]           Gemeint ist jeweils das Back-End. Damit sind alle Teile gemeint, die über den NodeJS-Server und nicht direkt im Browser der Benutzers laufen.
    %\item[GUI]              Grafische Benutzeroberfläche, Benutzerschnittstelle der Applikation
    %\item[MVP]              Ein Minimum Viable Product (MVP), ist die erste minimal funktionsfähige Iteration eines Produkts, das entwickelt werden muss, um mit minimalem Aufwand den Kunden-, Markt- oder Funktionsbedarf zu decken und handlungsrelevantes Feedback zu gewährleisten.
    %\item[PaaS]             Platform as a Service
\end{glossar}
