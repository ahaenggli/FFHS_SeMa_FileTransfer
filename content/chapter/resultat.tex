\chapter{Ausblick}
In Zukunft wird das in dieser Arbeit erstellte Konzept zu Testzwecken produktiv bei 100
IoT-Geräten eingesetzt. Die Evaluierung der Datenpunkte ist dabei noch nicht abgeschlossen
aber im Gange. Dabei werden die neuen Datenkorrelationen im Team diskutiert
und bewertet, die nach Abschluss dieses Verfahrens in die Firmware eingebaut
werden. Die Auswertung erfolgt dann in etwa drei Monaten nach der Aktualisierung. Dies
garantiert eine aussagekräftige Anzahl an Daten.
Um die Analysemöglichkeiten noch verbessern zu können, wird ausserdem MQTT als
neues Verbindungsprotokoll eingesetzt. Dieses Protokoll ermöglicht eine nahezu Realtime-
Analyse der gesammelten Daten, was eine geringere Reaktionszeit zur Folge hat. Diese
Strategie wird noch durch automatisierte Aktionen ergänzt, was den Arbeitsaufwand für
die Analyse minimal hält. Aus betrieblichen und datenschutzrechtlichen Gründen kann
die Strategie aber an dieser Stelle nicht näher ausgeführt werden.

\chapter{Reflexion}
Es handelt sich hierbei um meine erste wissenschaftliche Arbeit. 
Entsprechend musste ich mich vor Beginn stark in die Thematik einlesen. Dazu habe ich auch einige Beispielarbeiten gelesen.
Dies hat einiges an Zeit gekostet, wodurch ich mich viel zu spät damit auseinander gesetzt habe, 
über welches Thema ich schreiben soll. Das Coaching durch den Dozenten war daher sehr hilfreich. 
Während der Recherche zur dahinterliegenden Technologie der Arbeit, war es schwierig geeignete Quellen zu finden.
Die Themen werden in den Sachbüchern jeweils nur angeschnitten oder auf einem sehr hohen Expertenlevel sehr detailliert behandelt.
Das von mir gewünschte Mittelmass scheint leider nicht sehr weit verbreitet zu sein.
Dies kann an meiner Problemstellung zusammenliegen. Die zwar individuelle Lösung hat sehr allgemeingültige Grundlagen dahinter.
Netzwerke und deren Aufbau sind nicht gerade meine Stärke. 
Dennoch war es sehr interessant die Hintergründe dazu bei der Implementierung des Prototyps zu berücksichtigen.
\\
Ich bin froh darüber, dass es sich hierbei um eine erste wissenschaftliche Arbeit und keine Abschlussarbeit oder gar Bachelorarbeit handelt.
Mit den neu gewonnen Erfahrungen würde ich dort die Vorbereitungen und Recherche anders gestalten. 
Wäre mir von Beginn des Moduls an, ein Thema im Kopf gewesen, hätte ich die Arbeit wohl anders aufgebaut. 
Eine sinnvolle und möglichst vollständige Argumentation mit rotem Faden aufzubauen ist eine Herausforderung. 
Eventuell wäre es einfacher geworden, wenn die Fragestellung konkreter definiert worden wäre. 
Sie lässt eigentlich offen, ob der Prototyp nur ein Mittel zum Zweck oder aber der Entwurf dessen das Hauptziel ist. 
\\
Nichtsdestotrotz konnte an dieser Arbeit einiges an Erfahrung gewonnen werden. Mit den Resultaten bin ich im grossen und ganzen sehr zufrieden.
Bei der noch folgenden Bachelorarbeit werde ich jedoch die obigen Punkte verbessern. Besonders was die Definition der Fragestellung betrifft.
