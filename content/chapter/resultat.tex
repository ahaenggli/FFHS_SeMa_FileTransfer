\chapter{Ausblick}
Im Rahmen dieser Arbeit wurden aus zeitlichen Gründen nur wenige Eigenschaften definiert und implementiert.
Ebenfalls wurde die Sprache bewusst auf schweizerdeutsch gewählt als klare Kennzeichnung für die Testphase. 
Der Prototyp bietet eine gute Grundlage für weitere Ergänzungen. 
Bis zum produktiven Einsatz müssten noch folgende Punkte berücksichtigt werden:
\begin{itemize}
    \item Übersetzung auf Hochdeutsch
    \item Schnelleres Zusammensetzen von grossen Dateien (Start schon während dem Upload, nicht erst am Ende)
    \item Löschen von Upload- und Downloadlinks ohne Dateien zu löschen
    \item Erweiterte Aufzeichnung für zukünftig bessere Analysemöglichkeiten
\end{itemize}
Die aufgeführten Punkte sind nicht allzu aufwändig. So ist mit ca. 2-3h pro Punkt zu rechnen. 
Folglich ist der Prototyp bereits innert 1-2 Arbeitstagen für den produktiven Einsatz geeignet.
\\
Das Konzept bietet an sich sehr viel potential. 
Es besteht jedoch die Gefahr, dass die Anforderungen stetig wachsen könnten.
Wichtig ist daher, dass stets Alternativen miteinbezogen werden und jede zukünftige Anforderung vorab diskutiert wird.
Je mehr Anforderungen umgesetzt würden, je grösser ist die Wahrscheinlichkeit, 
dass ein bestehendes Tool die Bedürfnisse besser erfüllen könnte.

\chapter{Reflexion}
Es handelt sich hierbei um meine erste wissenschaftliche Arbeit. 
Entsprechend musste ich mich vor Beginn stark in die Thematik einlesen. Dazu habe ich auch einige Beispielarbeiten gelesen.
Dies hat einiges an Zeit gekostet, wodurch ich mich viel zu spät damit auseinander gesetzt habe, 
über welches Thema ich schreiben soll. Das Coaching durch den Dozenten war daher sehr hilfreich. 
Während der Recherche zur dahinterliegenden Technologie der Arbeit, war es schwierig geeignete Quellen zu finden.
Die Themen werden in den Sachbüchern jeweils nur angeschnitten oder auf einem sehr hohen Expertenlevel sehr detailliert behandelt.
Das von mir gewünschte Mittelmass scheint leider nicht sehr weit verbreitet zu sein.
Dies kann an meiner Problemstellung zusammenliegen. Die zwar individuelle Lösung hat sehr allgemeingültige Grundlagen dahinter.
Netzwerke und deren Aufbau sind nicht gerade meine Stärke. 
Dennoch war es sehr interessant die Hintergründe dazu bei der Implementierung des Prototyps zu berücksichtigen.
\\
Ich bin froh darüber, dass es sich hierbei um eine erste wissenschaftliche Arbeit und keine Abschlussarbeit oder gar Bachelorarbeit handelt.
Mit den neu gewonnen Erfahrungen würde ich dort die Vorbereitungen und Recherche anders gestalten. 
Wäre mir von Beginn des Moduls an, ein Thema im Kopf gewesen, hätte ich die Arbeit wohl anders aufgebaut. 
Eine sinnvolle und möglichst vollständige Argumentation mit rotem Faden aufzubauen ist eine Herausforderung. 
Eventuell wäre es einfacher geworden, wenn die Fragestellung konkreter definiert worden wäre. 
Sie lässt eigentlich offen, ob der Prototyp nur ein Mittel zum Zweck oder aber der Entwurf dessen das Hauptziel ist. 
\\
Nichtsdestotrotz konnte an dieser Arbeit einiges an Erfahrung gewonnen werden. Mit den Resultaten bin ich im grossen und ganzen sehr zufrieden.
Bei der noch folgenden Bachelorarbeit werde ich jedoch die obigen Punkte verbessern. Besonders was die Definition der Fragestellung betrifft.